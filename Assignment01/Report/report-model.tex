\documentclass[conference]{IEEEtran}

\usepackage[utf8]{inputenc}

\PassOptionsToPackage{bookmarks=false}{hyperref}

\usepackage{amsmath}
\usepackage{float}

\usepackage{listings}
\usepackage{color}
\usepackage{graphicx}
\usepackage{tikz}
\usepackage[caption=true,font=footnotesize]{subfig}
\usepackage[hidelinks]{hyperref}

\IEEEoverridecommandlockouts
% The preceding line is only needed to identify funding in the first footnote. If that is unneeded, please comment it out.
\usepackage[portuguese, english]{babel}
\usepackage{amsmath,amssymb,amsfonts}
\usepackage{algorithmic}
\usepackage{textcomp}
\def\BibTeX{{\rm B\kern-.05em{\sc i\kern-.025em b}\kern-.08em
    T\kern-.1667em\lower.7ex\hbox{E}\kern-.125emX}}
\begin{document}

\title{Estudos em Regressão Linear}

\author{\IEEEauthorblockN{Matheus Mortatti Diamantino}
\IEEEauthorblockA{
RA 156740 \\
matheusmortatti@gmail.com}
\and
\IEEEauthorblockN{Jos\'{e} Renato Vicente}
\IEEEauthorblockA{
RA 155984\\
joserenatovi@gmail.com}}

\maketitle

\section{Introduction}

Neste projeto, foi feito um estudo de predição de preço de diamantes. Para isso, foi utilizada uma base de dados com 54000 diamantes, em que são apresentados seus preços e nove features.

\section{Activities}

Regressão Linear é o método mais básico de Machine Learning utilizado para predizer o valor de uma variável dependente baseado em valores de variáveis independentes.

A regressão linear é chamada linear porque se considera que a relação da resposta às variáveis é uma função linear de alguns parâmetros.
Desta forma, dado um vetor Theta de tamanho igual ao número de features, cujo valor queremos determinar, temos que:

Preço_Alvo_Esperado = Theta * X

Em que X é um vetor com os valores das features para um dado diamante, cujo preço queremos determinar.

Para encontrar esse valor de Theta, utilizaremos alguns métodos e compararemos os resultados obtidos com cada um.

O primeiro método a ser analisado é o Gradient Descent. Neste método, utilizando nossa base de dados de treino, tentaremos encontrar o vetor Theta de modo que nosso preço alvo esperado fique o mais próximo possível do resultado real, cujo valor já sabemos.


\section{Proposed Solutions}
Normalização de Features:

Para entender melhor como cada feature influencia o preço de um diamante, plotamos nove gráficos de preço x valor da feature:

#### GRAFICOS ####

Analisando esses gráficos, podemos perceber que algumas features parecem ter maior influencia sobre o preço do que outras. Também podemos perceber que algumas features parecem ter influencia mais próxima de uma função quadrática do que linear sobre o preço do diamante. Na seção de experimentos, mostraremos o que foi feito com relação à essas diferenças.

Já que alguns valores das features encontram-se na casa das unidades e outros das centenas, decidimos fazer a Normalização das features para garantir resultados o amis precisos possśveis. Tal normalização foi feita da seguinte forma:

#### FORMULA ####



Regressão Linear:

Para a nossa primeira abordagem do problema, utilizamos uma regressão linear da forma: (FORMULA BASICA DA REGRESSAO LINEAR).
Para descidir que Learning Rate utilizar e quantas iterações deveríamos ter, calculamos o erro ao longo da execução do algoritmo pela fórmula:

#### FORMULA DO ERRO ####



\section{Experiments and Discussion}

Com o intuito de analisar o impacto de difentes algoritmos e inputs nas respostas fiansi, fizemos diversos experimentos.
Foram testados os seguintes algoritmos:
Gradient Descent nas versões Batch, Mini Batch e Stochastic
Normal Equation
Além disso, testamos os inputs:
Learning Rates
Parada por Numero de iterações e por Convergência




\section{Conclusions and Future Work}

The main conclusions of the work as well as some future directions for other people interested in continuing this work.

\bibliographystyle{IEEEtran}
\bibliography{biblio-link,biblio}

\end{document}
