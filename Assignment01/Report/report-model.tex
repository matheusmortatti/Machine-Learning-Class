\documentclass[conference]{IEEEtran}

\usepackage[utf8]{inputenc}

\PassOptionsToPackage{bookmarks=false}{hyperref}

\usepackage{amsmath}
\usepackage{float}

\usepackage{listings}
\usepackage{color}
\usepackage{graphicx}
\usepackage{tikz}
\usepackage[caption=true,font=footnotesize]{subfig}
\usepackage[hidelinks]{hyperref}

\IEEEoverridecommandlockouts
% The preceding line is only needed to identify funding in the first footnote. If that is unneeded, please comment it out.
\usepackage[portuges,brazil,english]{babel}
\usepackage{amsmath,amssymb,amsfonts}
\usepackage{algorithmic}
\usepackage{textcomp}
\def\BibTeX{{\rm B\kern-.05em{\sc i\kern-.025em b}\kern-.08em
    T\kern-.1667em\lower.7ex\hbox{E}\kern-.125emX}}
\begin{document}

\title{Report Title}

\author{\IEEEauthorblockN{Matheus Mortatti Diamantino}
\IEEEauthorblockA{
156740 \\
matheusmortatti@gmail.com}
\and
\IEEEauthorblockN{Jos\'{e} Renato Vicente}
\IEEEauthorblockA{
RA \\
email address}}

\maketitle

\section{Introduction}

The introduction and motivation of the work.

Some directions for the paper:

\begin{itemize}
	\item Figures and graphics are encouraged for making the
	report richer.
	\item {\bf The sections proposed here are not hard-constrained}. It means you can propose other sections as well as change the existing ones.
	\item Please number citations consecutively within brackets~\cite{b1}.
\end{itemize}

\section{Activities}

The state-of-the-art research (about prior work for solving the same problem).

\section{Proposed Solutions}

The proposed solutions for the selected problem.

\section{Experiments and Discussion}

The experiments carried out and the obtained
results.

\section{Conclusions and Future Work}

The main conclusions of the work as well as some future directions for other people interested in continuing this work.

\bibliographystyle{IEEEtran}
\bibliography{biblio-link,biblio}

\end{document}
